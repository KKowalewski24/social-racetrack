\documentclass[../Kamil_Kowalewski_Main.tex]{subfiles}

\begin{document} {

    W erze potężnego rozwoju cyfrowego świata portale społecznościowe stały się
    nieodłączną częścią życia ogromnego odsetka społeczeństwa na naszym globie. Ich
    cele i~założenia są różne, jednak wspólnym celem jest łącznie określonych grup
    społecznych. Aplikacja społecznościowa \textit{Social Racetrack} ma za zadanie
    spajać i~przyciągać osoby zainteresowane sportami samochodowymi.

    \section{Problematyka}
    \label{chapter1:wprowadzenie_sporty:problematyka} {
        Problematyka związana z tworzeniem społeczności jest bardzo rozległa
        i~zdecydowanie można o niej powiedzieć, że nie jest trywialna. Jej podwaliny
        można było zaobserwować już setki, jeśli nie tysiące lat temu w pierwotnych ludach
        zamieszkujących nasz glob. Z~biegiem czasu oraz rozwojem kulturowym zmieniały
        się relacje, zależności oraz tematyka czy też cel skupiający dane grupy ludzi.

        Aktualnie poprzez bardzo szybki rozwój technologii a w szczególności
        niesamowity postęp w~informatyzacji społeczności zaczęły się przenosić do
        globalnej sieci Internet, która poprzez portale internetowe, często potocznie
        określane jako strony internetowe, takie jak Facebook\cite{website:facebook}
        czy Twitter\cite{website:twitter}, łączą ogromne liczby osób w niesamowitej,
        kiedyś nie do pomyślenia liczbie społeczności. Społeczności te są zgromadzone
        wokół bardzo różnych dziedzin takich jak np. nauka czy sport stąd mają bardzo
        zróżnicowane potrzeby. Wcześniej wymienione portale starają się sprostać tym
        wymaganiom lecz poprzez dużą rozpiętość dziedzin jest to niezwykle trudne.

        Dosyć wyjątkową niszą są grupy zrzeszające miłośników sportów samochodowych,
        gdyż zazwyczaj są to osoby o wysokim statusie społecznym, mające relatywnie
        wysokie wymagania. Kolejnym aspektem jest fakt, iż przepisy związane z~sportami
        samochodowymi są mocno rygorystyczne więc dostosowanie ogólnych portali
        społecznościowych do tych potrzeb mogłoby wpłynąć negatywnie na inne grupy
        społeczne co najprawdopodobniej skutkowałoby rezygnacją z użycia przez coraz
        większe liczby ludzi a~finalnie mogłoby się skończyć na potężnej utracie
        popularności danego portalu i~jego likwidacji gdyż nie byłby on rentowny.
    }

    \section{Cel i założenia projektu}
    \label{chapter1:wprowadzenie_sporty:cel} {
        Celem niniejszej pracy inżynierskiej jest zaimplementowanie aplikacji
        społecznościowej oraz udostępnienie jej w~formie nieodpłatnej osobom związanym
        ze sportami samochodowymi oraz tym osobom, które chcą rozpocząć przygodę z~tym
        sportem. W zakres pracy można włączyć zarówno analizy aktualnie dostępnych
        rozwiązań na rynku, jak i~przygotowanie aplikacji serwerowej z~wykorzystaniem
        platformy chmurowej Firebase oraz aplikacji dla klienta z~użyciem technologii
        Node.js oraz React.
    }

    \section{Układ pracy}
    \label{chapter1:wprowadzenie_sporty:uklad} {
        Rozdział pierwszy stanowi wstęp do problemu zaobserwowanego przez autora pracy.
        Drugi rozdział prezentuję sporty samochodowe, aktualnie dostępne aplikacje
        społecznościowe i~ich analizę oraz uzasadnienie stworzenie aplikacji
        \textit{Social Racetrack}. Kolejnym rozdziałem jest rozdział trzeci,
        którego zadaniem jest przedstawienie technologii wykorzystanych w aplikacji.
        Rozdział czwarty skupia się na dokumentacji technicznej składającej się~z
        wymagań funkcjonalnych oraz niefunkcjonalnych, architektury aplikacji, jej
        implementacji oraz procesu instalacji. Następnym, piątym z~kolei rozdziałem
        jest dokumentacja użytkownika tworzonej aplikacji. Ostatni rozdział
        podsumowuje całą pracę i~jest zwieńczeniem przedstawionego problemu oraz jego
        rozwiązania.
    }

}
\end{document}
