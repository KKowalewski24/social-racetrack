\documentclass[../Kamil_Kowalewski_Main.tex]{subfiles}

\begin{document} {

    Zadaniem pierwszej części niniejszej pracy inżynierskiej było wprowadzenie do
    tematyki związanej z~sportami samochodowymi, w której została przedstawiona ich
    historia i~rozwój na przestrzeni lat. Następnie został zaprezentowany podział na
    kategorie w~tej dziedzinie sportu oraz zostały przedstawione bardzo znane obiekty
    sportowe znajdujące się w~Europie oraz amatorskie starty w~zawodach.
    W~kolejnych sekcjach zostały przedstawione wady aktualnie dostępnych aplikacji
    społecznościowych, które pozostawiły lukę wypełnioną przez aplikacje
    \textit{Social Racetrack}.

    W~kolejnej części zostały przedstawione technologie użyte w czasie tworzenia
    aplikacji. Dobór ich nie był pozostawiony przypadkowi. Były one wybrane ze względu
    na dużą popularność, wsparcie oraz kompatybilność wsteczną, dzięki czemu aplikacja
    będzie mogła być rozszerzana o~kolejne funkcjonalność bez potrzeby jej gruntownej
    przebudowy. Samo jej utrzymanie nie będzie stanowić problemu.

    Trzecia już z~kolei część zawiera dokumentację techniczną aplikacji. Zostały tam
    przedstawione wymagania funkcjonalne i~niefunkcjonalne oraz sam opis implementacji
    i~procesu instalacji. Tak jak zostało wcześniej wspomniane aplikacja została
    stworzona z~myślą o możliwości rozszerzania jej funkcjonalności. Została ona
    zbudowana w~sposób modularny przy pomocy podziału na komponenty w~bibliotece React.
    Sam podział na warstwy jest bardzo ważny. Zapewnia on czytelny i~łatwy przepływ
    danych przez aplikacje oraz elementy odpowiedzialne za określone czynności są
    zgrupowane w~wybranych miejscach. Sam proces instalacji został maksymalnie
    uproszczony poprzez zastosowanie wszystkich dostępnych narzędzi platformy chmurowej
    Firebase oraz lokalnych skryptów.

    Ostatnia część ukazuje pełną dokumentację użytkownika, dzięki której rozpoczęcie
    korzystania z~aplikacji jest niezwykle proste. Dodatkowo minimalistyczny interfejs
    użytkownika z~wykorzystaniem Material Design wspiera szybką naukę obsługi aplikacji.

    Wedle szacunków dotyczących rozwoju portali społecznościowych, aplikacja
    \textit{Social Racetrack} w~przeciągu najbliższych kilku miesięcy powinna zgromadzić
    wiele jej aktywnych użytkowników, którzy są miłośnikami sportów samochodowych
    i~wypełnić lukę w~portalach społecznościowych zauważoną przez autora pracy dyplomowej.
}
\end{document}
